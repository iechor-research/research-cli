\documentclass[11pt,a4paper]{article}
\usepackage[UTF8]{ctex}
\usepackage{amsmath,amssymb,amsthm}
\usepackage{graphicx}
\usepackage{hyperref}
\usepackage{geometry}
\usepackage{xcolor}
\usepackage{tikz}

% 页面设置
\geometry{margin=2.5cm}
\hypersetup{colorlinks=true, linkcolor=blue, citecolor=red}

% 自定义命令
\newcommand{\qianwen}{\textcolor{blue}{\textbf{Qianwen}}}

\title{\qianwen\ 模式下的 LaTeX 学术写作}
\author{
    Qianwen AI 研究团队 \\
    \texttt{research@qianwen.ai}
}
\date{\today}

\begin{document}

\maketitle

\begin{abstract}
本文演示了在 \qianwen\ 模式下进行 LaTeX 学术写作的完整流程。我们展示了如何使用 research-cli 工具创建、管理和编译 LaTeX 项目,特别关注中文学术写作的需求和最佳实践。
\end{abstract}

\tableofcontents

\section{引言}

\qianwen\ 是一个强大的大语言模型,在学术写作和 LaTeX 文档处理方面具有独特优势。本文档展示了如何充分利用这些功能。

\subsection{主要特性}

\begin{itemize}
    \item 中英文混合排版支持
    \item 智能模板选择和自定义
    \item 自动化编译和错误诊断
    \item 数学公式智能处理
\end{itemize}

\section{数学公式示例}

\subsection{基础公式}

爱因斯坦质能方程:
\begin{equation}
    E = mc^2
    \label{eq:einstein}
\end{equation}

\subsection{复杂数学表达式}

机器学习中的损失函数:
\begin{align}
    \mathcal{L}(\theta) &= \frac{1}{n} \sum_{i=1}^{n} \ell(f_\theta(x_i), y_i) \\
    &= \frac{1}{n} \sum_{i=1}^{n} -\log p(y_i | x_i; \theta)
\end{align}

注意力机制计算:
\begin{equation}
    \text{Attention}(Q, K, V) = \text{softmax}\left(\frac{QK^T}{\sqrt{d_k}}\right)V
\end{equation}

\section{图表支持}

\subsection{简单图形}

\begin{figure}[h]
\centering
\begin{tikzpicture}
    \draw[->] (0,0) -- (4,0) node[right] {$x$};
    \draw[->] (0,0) -- (0,3) node[above] {$y$};
    \draw[domain=0:3.5, smooth, variable=\x, blue, thick] 
        plot ({\x}, {\x*\x/4});
    \node at (2, 2.5) {$y = \frac{x^2}{4}$};
\end{tikzpicture}
\caption{\qianwen\ 生成的数学函数图像}
\label{fig:parabola}
\end{figure}

\section{代码展示}

\qianwen\ 支持代码块的智能格式化:

\begin{verbatim}
def qianwen_latex_demo():
    """Qianwen LaTeX 演示函数"""
    print("欢迎使用 Qianwen 模式!")
    return "LaTeX 项目创建成功"
\end{verbatim}

\section{表格示例}

\begin{table}[h]
\centering
\begin{tabular}{|c|c|c|}
\hline
\textbf{功能} & \textbf{支持程度} & \textbf{备注} \\
\hline
中文排版 & 完全支持 & 使用 ctex 宏包 \\
数学公式 & 完全支持 & amsmath 增强 \\
图片插入 & 完全支持 & graphicx 宏包 \\
参考文献 & 完全支持 & BibTeX 集成 \\
\hline
\end{tabular}
\caption{\qianwen\ LaTeX 功能支持情况}
\label{tab:features}
\end{table}

\section{最佳实践}

\subsection{中文学术写作建议}

\begin{enumerate}
    \item 使用 \texttt{xelatex} 编译器处理中文字体
    \item 合理设置页面边距和行距
    \item 使用语义化的章节结构
    \item 保持数学符号的一致性
\end{enumerate}

\subsection{\qianwen\ 模式优势}

\begin{itemize}
    \item 智能错误检测和修复建议
    \item 自动化的格式优化
    \item 上下文感知的内容生成
    \item 多语言混排支持
\end{itemize}

\section{结论}

本文展示了 \qianwen\ 模式在 LaTeX 学术写作中的强大功能。通过 research-cli 工具,用户可以:

\begin{itemize}
    \item 快速创建专业的学术文档
    \item 享受智能化的编译和诊断服务
    \item 获得中英文混合排版的完美支持
    \item 利用 AI 辅助进行内容创作和格式优化
\end{itemize}

未来,我们将继续改进 \qianwen\ 的 LaTeX 支持,为学术研究者提供更加便捷和强大的写作工具。

\section*{致谢}

感谢 \qianwen\ 团队在自然语言处理和文档生成方面的杰出贡献。

% 参考文献
\begin{thebibliography}{9}
\bibitem{latex}
Leslie Lamport. \textit{LaTeX: A Document Preparation System}. Addison-Wesley, 1994.

\bibitem{qianwen}
Qianwen Team. \textit{Qianwen: A Large Language Model for Academic Writing}. arXiv preprint, 2024.

\bibitem{ctex}
ctex-kit Team. \textit{ctex: LaTeX Classes and Packages for Chinese Typesetting}. CTAN, 2023.
\end{thebibliography}

\end{document}